% Options for packages loaded elsewhere
\PassOptionsToPackage{unicode}{hyperref}
\PassOptionsToPackage{hyphens}{url}
%
\documentclass[
]{article}
\usepackage{amsmath,amssymb}
\usepackage{lmodern}
\usepackage{iftex}
\ifPDFTeX
  \usepackage[T1]{fontenc}
  \usepackage[utf8]{inputenc}
  \usepackage{textcomp} % provide euro and other symbols
\else % if luatex or xetex
  \usepackage{unicode-math}
  \defaultfontfeatures{Scale=MatchLowercase}
  \defaultfontfeatures[\rmfamily]{Ligatures=TeX,Scale=1}
\fi
% Use upquote if available, for straight quotes in verbatim environments
\IfFileExists{upquote.sty}{\usepackage{upquote}}{}
\IfFileExists{microtype.sty}{% use microtype if available
  \usepackage[]{microtype}
  \UseMicrotypeSet[protrusion]{basicmath} % disable protrusion for tt fonts
}{}
\makeatletter
\@ifundefined{KOMAClassName}{% if non-KOMA class
  \IfFileExists{parskip.sty}{%
    \usepackage{parskip}
  }{% else
    \setlength{\parindent}{0pt}
    \setlength{\parskip}{6pt plus 2pt minus 1pt}}
}{% if KOMA class
  \KOMAoptions{parskip=half}}
\makeatother
\usepackage{xcolor}
\usepackage[margin=1in]{geometry}
\usepackage{color}
\usepackage{fancyvrb}
\newcommand{\VerbBar}{|}
\newcommand{\VERB}{\Verb[commandchars=\\\{\}]}
\DefineVerbatimEnvironment{Highlighting}{Verbatim}{commandchars=\\\{\}}
% Add ',fontsize=\small' for more characters per line
\usepackage{framed}
\definecolor{shadecolor}{RGB}{248,248,248}
\newenvironment{Shaded}{\begin{snugshade}}{\end{snugshade}}
\newcommand{\AlertTok}[1]{\textcolor[rgb]{0.94,0.16,0.16}{#1}}
\newcommand{\AnnotationTok}[1]{\textcolor[rgb]{0.56,0.35,0.01}{\textbf{\textit{#1}}}}
\newcommand{\AttributeTok}[1]{\textcolor[rgb]{0.77,0.63,0.00}{#1}}
\newcommand{\BaseNTok}[1]{\textcolor[rgb]{0.00,0.00,0.81}{#1}}
\newcommand{\BuiltInTok}[1]{#1}
\newcommand{\CharTok}[1]{\textcolor[rgb]{0.31,0.60,0.02}{#1}}
\newcommand{\CommentTok}[1]{\textcolor[rgb]{0.56,0.35,0.01}{\textit{#1}}}
\newcommand{\CommentVarTok}[1]{\textcolor[rgb]{0.56,0.35,0.01}{\textbf{\textit{#1}}}}
\newcommand{\ConstantTok}[1]{\textcolor[rgb]{0.00,0.00,0.00}{#1}}
\newcommand{\ControlFlowTok}[1]{\textcolor[rgb]{0.13,0.29,0.53}{\textbf{#1}}}
\newcommand{\DataTypeTok}[1]{\textcolor[rgb]{0.13,0.29,0.53}{#1}}
\newcommand{\DecValTok}[1]{\textcolor[rgb]{0.00,0.00,0.81}{#1}}
\newcommand{\DocumentationTok}[1]{\textcolor[rgb]{0.56,0.35,0.01}{\textbf{\textit{#1}}}}
\newcommand{\ErrorTok}[1]{\textcolor[rgb]{0.64,0.00,0.00}{\textbf{#1}}}
\newcommand{\ExtensionTok}[1]{#1}
\newcommand{\FloatTok}[1]{\textcolor[rgb]{0.00,0.00,0.81}{#1}}
\newcommand{\FunctionTok}[1]{\textcolor[rgb]{0.00,0.00,0.00}{#1}}
\newcommand{\ImportTok}[1]{#1}
\newcommand{\InformationTok}[1]{\textcolor[rgb]{0.56,0.35,0.01}{\textbf{\textit{#1}}}}
\newcommand{\KeywordTok}[1]{\textcolor[rgb]{0.13,0.29,0.53}{\textbf{#1}}}
\newcommand{\NormalTok}[1]{#1}
\newcommand{\OperatorTok}[1]{\textcolor[rgb]{0.81,0.36,0.00}{\textbf{#1}}}
\newcommand{\OtherTok}[1]{\textcolor[rgb]{0.56,0.35,0.01}{#1}}
\newcommand{\PreprocessorTok}[1]{\textcolor[rgb]{0.56,0.35,0.01}{\textit{#1}}}
\newcommand{\RegionMarkerTok}[1]{#1}
\newcommand{\SpecialCharTok}[1]{\textcolor[rgb]{0.00,0.00,0.00}{#1}}
\newcommand{\SpecialStringTok}[1]{\textcolor[rgb]{0.31,0.60,0.02}{#1}}
\newcommand{\StringTok}[1]{\textcolor[rgb]{0.31,0.60,0.02}{#1}}
\newcommand{\VariableTok}[1]{\textcolor[rgb]{0.00,0.00,0.00}{#1}}
\newcommand{\VerbatimStringTok}[1]{\textcolor[rgb]{0.31,0.60,0.02}{#1}}
\newcommand{\WarningTok}[1]{\textcolor[rgb]{0.56,0.35,0.01}{\textbf{\textit{#1}}}}
\usepackage{graphicx}
\makeatletter
\def\maxwidth{\ifdim\Gin@nat@width>\linewidth\linewidth\else\Gin@nat@width\fi}
\def\maxheight{\ifdim\Gin@nat@height>\textheight\textheight\else\Gin@nat@height\fi}
\makeatother
% Scale images if necessary, so that they will not overflow the page
% margins by default, and it is still possible to overwrite the defaults
% using explicit options in \includegraphics[width, height, ...]{}
\setkeys{Gin}{width=\maxwidth,height=\maxheight,keepaspectratio}
% Set default figure placement to htbp
\makeatletter
\def\fps@figure{htbp}
\makeatother
\setlength{\emergencystretch}{3em} % prevent overfull lines
\providecommand{\tightlist}{%
  \setlength{\itemsep}{0pt}\setlength{\parskip}{0pt}}
\setcounter{secnumdepth}{-\maxdimen} % remove section numbering
\ifLuaTeX
  \usepackage{selnolig}  % disable illegal ligatures
\fi
\IfFileExists{bookmark.sty}{\usepackage{bookmark}}{\usepackage{hyperref}}
\IfFileExists{xurl.sty}{\usepackage{xurl}}{} % add URL line breaks if available
\urlstyle{same} % disable monospaced font for URLs
\hypersetup{
  pdftitle={STT380 HW1},
  pdfauthor={Jimmy Gray-Jones},
  hidelinks,
  pdfcreator={LaTeX via pandoc}}

\title{STT380 HW1}
\author{Jimmy Gray-Jones}
\date{2022-09-06}

\begin{document}
\maketitle

\#Question 1

\#For 2 events A and B, the probability of A occurring is 0.50, and the
probability of B occurring is 0.4. The probability of neither A nor B
occurring is 0.4.

\#a. (a) What is the probability of A or B occurring?

\begin{Shaded}
\begin{Highlighting}[]
\NormalTok{.}\DecValTok{90} 
\end{Highlighting}
\end{Shaded}

\begin{verbatim}
## [1] 0.9
\end{verbatim}

\begin{Shaded}
\begin{Highlighting}[]
\CommentTok{\#(.4+.5)}
\end{Highlighting}
\end{Shaded}

\#b. (b) What is the probability of A and B occurring?

\begin{Shaded}
\begin{Highlighting}[]
\FloatTok{0.2} 
\end{Highlighting}
\end{Shaded}

\begin{verbatim}
## [1] 0.2
\end{verbatim}

\begin{Shaded}
\begin{Highlighting}[]
\CommentTok{\#(.4*.5)}
\end{Highlighting}
\end{Shaded}

\#Question 2

\#2 6-sided dice are rolled (each with values 1, 2, 3, 4, 5, 6). The
outcome of the roll is found by the difference between the larger and
smaller numbers (so that if a 3 and 5 are rolled, the result is a 2, if
a 5 and a 1 is rolled, the results is a 4, if the results is a 3 and a
3, the results is a 0, etc.)

\#a. Find the sample space.

\begin{Shaded}
\begin{Highlighting}[]
\CommentTok{\#The Sample Space is 36}

\DocumentationTok{\#\#(We have two 6 sided dice. Each dice roll can be anything from \{1,6\} to \{6,6\}. To find the sample space, we multiply the amount of outcomes from each dice (6*6))}
\end{Highlighting}
\end{Shaded}

\#b. Find the probability that the result is a 1.

\begin{Shaded}
\begin{Highlighting}[]
\DecValTok{10}\SpecialCharTok{/}\DecValTok{36}
\end{Highlighting}
\end{Shaded}

\begin{verbatim}
## [1] 0.2777778
\end{verbatim}

\begin{Shaded}
\begin{Highlighting}[]
\CommentTok{\#Given the rules of the game, a 1 can only be ever made if the dice create the outcomes }
\CommentTok{\#\{1,2\},\{2,3\},\{3,4\},\{4,5\},\{5,6\}, and vice versa.}
\CommentTok{\#That means out of 36 potential outcomes, 10 of them can result in a 1.}
\end{Highlighting}
\end{Shaded}

\#c.~Create a simulation of 10,000 rolls of the 2 dice. Calculate the
difference, and find the proportion of rolls for which the result is a
1. Does your result approximately agree with what you got in (b)?

\begin{Shaded}
\begin{Highlighting}[]
\CommentTok{\#Creating dice vector}
\NormalTok{dice }\OtherTok{=} \FunctionTok{c}\NormalTok{(}\DecValTok{1}\NormalTok{,}\DecValTok{2}\NormalTok{,}\DecValTok{3}\NormalTok{,}\DecValTok{4}\NormalTok{,}\DecValTok{5}\NormalTok{,}\DecValTok{6}\NormalTok{)}

\CommentTok{\#setting rolls of two different dice, 10000 times}
\NormalTok{a\_dice }\OtherTok{=} \FunctionTok{sample}\NormalTok{(}\AttributeTok{x=}\NormalTok{dice,}\AttributeTok{size=}\DecValTok{10000}\NormalTok{,}\AttributeTok{replace=}\ConstantTok{TRUE}\NormalTok{)}

\NormalTok{b\_dice }\OtherTok{=} \FunctionTok{sample}\NormalTok{(}\AttributeTok{x=}\NormalTok{dice,}\AttributeTok{size=}\DecValTok{10000}\NormalTok{,}\AttributeTok{replace=}\ConstantTok{TRUE}\NormalTok{)}

\NormalTok{outcome }\OtherTok{=} \FunctionTok{c}\NormalTok{()}

\ControlFlowTok{for}\NormalTok{(i }\ControlFlowTok{in} \DecValTok{1}\SpecialCharTok{:}\DecValTok{10000}\NormalTok{)}
\NormalTok{\{}
  \ControlFlowTok{if}\NormalTok{(a\_dice[i]}\SpecialCharTok{\textgreater{}}\NormalTok{b\_dice[i])}
\NormalTok{  \{}
\NormalTok{    outcome[i] }\OtherTok{=}\NormalTok{ a\_dice[i] }\SpecialCharTok{{-}}\NormalTok{ b\_dice[i]}
\NormalTok{  \}}
    \ControlFlowTok{if}\NormalTok{(a\_dice[i]}\SpecialCharTok{\textless{}}\NormalTok{b\_dice[i])}
\NormalTok{  \{}
\NormalTok{    outcome[i] }\OtherTok{=}\NormalTok{ b\_dice[i] }\SpecialCharTok{{-}}\NormalTok{ a\_dice[i]}
\NormalTok{  \}}
  \ControlFlowTok{if}\NormalTok{(a\_dice[i]}\SpecialCharTok{==}\NormalTok{b\_dice[i])}
\NormalTok{  \{}
\NormalTok{    outcome[i] }\OtherTok{=} \DecValTok{0}
\NormalTok{  \}}
\NormalTok{\}}

\NormalTok{count }\OtherTok{=} \DecValTok{0}

\ControlFlowTok{for}\NormalTok{(i }\ControlFlowTok{in} \DecValTok{1}\SpecialCharTok{:}\DecValTok{10000}\NormalTok{)}
\NormalTok{\{}
  \ControlFlowTok{if}\NormalTok{(outcome[i]}\SpecialCharTok{==}\DecValTok{1}\NormalTok{)}
\NormalTok{  \{}
\NormalTok{    count }\OtherTok{=}\NormalTok{ count }\SpecialCharTok{+} \DecValTok{1}
\NormalTok{  \}}
\NormalTok{\}}

\FunctionTok{print}\NormalTok{(}\StringTok{"Number of rolls that were a 1:"}\NormalTok{)}
\end{Highlighting}
\end{Shaded}

\begin{verbatim}
## [1] "Number of rolls that were a 1:"
\end{verbatim}

\begin{Shaded}
\begin{Highlighting}[]
\FunctionTok{print}\NormalTok{(count)}
\end{Highlighting}
\end{Shaded}

\begin{verbatim}
## [1] 2742
\end{verbatim}

\begin{Shaded}
\begin{Highlighting}[]
\FunctionTok{print}\NormalTok{(}\StringTok{"Proportion of rolls that were 1:"}\NormalTok{)}
\end{Highlighting}
\end{Shaded}

\begin{verbatim}
## [1] "Proportion of rolls that were 1:"
\end{verbatim}

\begin{Shaded}
\begin{Highlighting}[]
\FunctionTok{print}\NormalTok{(count}\SpecialCharTok{/}\DecValTok{10000}\NormalTok{)}
\end{Highlighting}
\end{Shaded}

\begin{verbatim}
## [1] 0.2742
\end{verbatim}

\begin{Shaded}
\begin{Highlighting}[]
\CommentTok{\#Result does not agree wth what I got in b}
\end{Highlighting}
\end{Shaded}

\#In a certain game, 1 6-sided die is rolled and 2 coins are flipped. A
person will win if the die rolls exactly the same value as the number of
heads flipped.

\#a. What is the probability of winning the game?

\begin{Shaded}
\begin{Highlighting}[]
\CommentTok{\# If 1 head is flipped, then the odds of winning are 1/12}

\CommentTok{\#(1/6 chance of getting \#1, and 1/2 chance of flipping heads)}

\CommentTok{\# If 2 heads are flipped, then the odds of winning are 1/24}

\CommentTok{\#(1/6 chance of getting \#2, and 1/4 chance of flipping two heads)}
\end{Highlighting}
\end{Shaded}

\#b. Create a simulation of 10,000 plays of the game. Does the number of
wins approximately agree with (a)?

\#Basically yes; The stats are roughly close, give or take a percentage
\#(Simulation below)

\begin{Shaded}
\begin{Highlighting}[]
\NormalTok{c\_dice }\OtherTok{=} \FunctionTok{sample}\NormalTok{(}\AttributeTok{x=}\NormalTok{dice,}\AttributeTok{size=}\DecValTok{10000}\NormalTok{,}\AttributeTok{replace=}\ConstantTok{TRUE}\NormalTok{)}
\NormalTok{coin\_1 }\OtherTok{=} \FunctionTok{sample}\NormalTok{(}\FunctionTok{c}\NormalTok{(}\DecValTok{1}\NormalTok{,}\DecValTok{2}\NormalTok{),}\AttributeTok{size=}\DecValTok{10000}\NormalTok{,}\AttributeTok{replace=}\ConstantTok{TRUE}\NormalTok{)}
\NormalTok{coin\_2 }\OtherTok{=} \FunctionTok{sample}\NormalTok{(}\FunctionTok{c}\NormalTok{(}\DecValTok{1}\NormalTok{,}\DecValTok{2}\NormalTok{),}\AttributeTok{size=}\DecValTok{10000}\NormalTok{,}\AttributeTok{replace=}\ConstantTok{TRUE}\NormalTok{)}

\CommentTok{\#1 will represent heads. 2 will represent tails, for the coins}

\NormalTok{x}\OtherTok{=}\DecValTok{0} \CommentTok{\#x is the count of single heads}
\NormalTok{y}\OtherTok{=}\DecValTok{0} \CommentTok{\#y is the count of two heads being flipped}

\ControlFlowTok{for}\NormalTok{(i }\ControlFlowTok{in} \DecValTok{1}\SpecialCharTok{:}\DecValTok{10000}\NormalTok{)}
\NormalTok{\{}
  \ControlFlowTok{if}\NormalTok{(c\_dice[i]}\SpecialCharTok{==}\DecValTok{1} \SpecialCharTok{\&}\NormalTok{ coin\_1[i]}\SpecialCharTok{==}\DecValTok{1}\NormalTok{)}
\NormalTok{  \{}
\NormalTok{    x }\OtherTok{=}\NormalTok{ x}\SpecialCharTok{+}\DecValTok{1}
\NormalTok{  \}}
  \ControlFlowTok{if}\NormalTok{(c\_dice[i]}\SpecialCharTok{==}\DecValTok{1} \SpecialCharTok{\&}\NormalTok{ coin\_2[i]}\SpecialCharTok{==}\DecValTok{1}\NormalTok{)}
\NormalTok{  \{}
\NormalTok{    x }\OtherTok{=}\NormalTok{ x}\SpecialCharTok{+}\DecValTok{1}
\NormalTok{  \}}
    \ControlFlowTok{if}\NormalTok{(c\_dice[i]}\SpecialCharTok{==}\DecValTok{1} \SpecialCharTok{\&}\NormalTok{ coin\_1[i]}\SpecialCharTok{==}\DecValTok{1} \SpecialCharTok{\&}\NormalTok{ coin\_2[i]}\SpecialCharTok{==}\DecValTok{1}\NormalTok{)}
\NormalTok{  \{}
\NormalTok{    y }\OtherTok{=}\NormalTok{ y}\SpecialCharTok{+}\DecValTok{1}
\NormalTok{  \}}
\NormalTok{\}}

\CommentTok{\#Represents if only one coin was heads}
\FunctionTok{print}\NormalTok{(x)}
\end{Highlighting}
\end{Shaded}

\begin{verbatim}
## [1] 1571
\end{verbatim}

\begin{Shaded}
\begin{Highlighting}[]
\CommentTok{\#Represents if both coins were heads}
\FunctionTok{print}\NormalTok{(y)}
\end{Highlighting}
\end{Shaded}

\begin{verbatim}
## [1] 375
\end{verbatim}

\begin{Shaded}
\begin{Highlighting}[]
\CommentTok{\#Finding out the percentage of a single heads vs. two heads}
\FunctionTok{print}\NormalTok{(y}\SpecialCharTok{/}\NormalTok{x)}
\end{Highlighting}
\end{Shaded}

\begin{verbatim}
## [1] 0.2387015
\end{verbatim}

\#Exercise 1.7 // Question \#4

\#In one game, de Mere bet that in four independent rolls of a fair die,
at least one six would appear. We will start by simulating that game

\#a) Write a function deMere1 that simulates one play of the game,
returning 1 if de Mere wins (i.e.~if at least one six appears on four
the four rolls) and a 0 if de Mere loses. You will probably find the
sample() function useful.

\begin{Shaded}
\begin{Highlighting}[]
\NormalTok{deMere1 }\OtherTok{=} \ControlFlowTok{function}\NormalTok{(numRep)}
\NormalTok{\{}
\NormalTok{  dem\_sim }\OtherTok{=} \FunctionTok{sample}\NormalTok{(}\FunctionTok{c}\NormalTok{(}\DecValTok{0}\NormalTok{,}\DecValTok{1}\NormalTok{),}\AttributeTok{size=}\NormalTok{numRep,}\AttributeTok{replace=}\ConstantTok{TRUE}\NormalTok{,}\AttributeTok{prob=}\FunctionTok{c}\NormalTok{(}\DecValTok{5}\SpecialCharTok{/}\DecValTok{6}\NormalTok{,}\DecValTok{1}\SpecialCharTok{/}\DecValTok{6}\NormalTok{))}
  
  \ControlFlowTok{for}\NormalTok{(i }\ControlFlowTok{in} \DecValTok{1}\SpecialCharTok{:}\NormalTok{numRep)}
\NormalTok{  \{}
    \ControlFlowTok{if}\NormalTok{(dem\_sim[i]}\SpecialCharTok{==}\DecValTok{1}\NormalTok{)}
\NormalTok{    \{}
      \FunctionTok{print}\NormalTok{(}\StringTok{"Dice rolls are:"}\NormalTok{)}
      \FunctionTok{print}\NormalTok{(dem\_sim)}
      \FunctionTok{return}\NormalTok{(}\DecValTok{1}\NormalTok{)}
\NormalTok{    \}}
     \ControlFlowTok{else} \ControlFlowTok{if}\NormalTok{(dem\_sim[i]}\SpecialCharTok{!=}\DecValTok{1} \SpecialCharTok{\&}\NormalTok{ i}\SpecialCharTok{==}\NormalTok{numRep)}
\NormalTok{    \{}
      \FunctionTok{print}\NormalTok{(}\StringTok{"Dice rolls are:"}\NormalTok{)}
      \FunctionTok{print}\NormalTok{(dem\_sim)}
      \FunctionTok{return}\NormalTok{(}\DecValTok{0}\NormalTok{)}
\NormalTok{    \}}
\NormalTok{  \}}
  
\NormalTok{\}}
\end{Highlighting}
\end{Shaded}

\begin{Shaded}
\begin{Highlighting}[]
\FunctionTok{deMere1}\NormalTok{(}\DecValTok{4}\NormalTok{)}
\end{Highlighting}
\end{Shaded}

\begin{verbatim}
## [1] "Dice rolls are:"
## [1] 1 0 0 1
\end{verbatim}

\begin{verbatim}
## [1] 1
\end{verbatim}

\#b)Now write a function called deMere1WinProb that takes as input a
number of repetitions numRep and which simulates numRep plays of the
game, returning the proportion in which de Mere wins. Run this function
with numRep equal to 50000 and report the proportion of games in which
de Mere wins. Is the proportion more or less than 0.5?

\begin{Shaded}
\begin{Highlighting}[]
\NormalTok{deMere1WinProb }\OtherTok{=} \ControlFlowTok{function}\NormalTok{(numRep)}
\NormalTok{\{}
\NormalTok{  dem\_dice }\OtherTok{=} \FunctionTok{c}\NormalTok{(}\DecValTok{1}\NormalTok{,}\DecValTok{2}\NormalTok{,}\DecValTok{3}\NormalTok{,}\DecValTok{4}\NormalTok{,}\DecValTok{5}\NormalTok{,}\DecValTok{6}\NormalTok{)}
\NormalTok{  sim }\OtherTok{=} \FunctionTok{sample}\NormalTok{(dem\_dice, }\AttributeTok{size=}\NormalTok{numRep,}\AttributeTok{replace=}\ConstantTok{TRUE}\NormalTok{)}
  
\NormalTok{  wins}\OtherTok{=}\DecValTok{0}
\NormalTok{  loss}\OtherTok{=}\DecValTok{0}
  
  \ControlFlowTok{for}\NormalTok{(i }\ControlFlowTok{in} \DecValTok{1}\SpecialCharTok{:}\NormalTok{numRep)}
\NormalTok{  \{}
    \ControlFlowTok{if}\NormalTok{(sim[i]}\SpecialCharTok{==}\DecValTok{6}\NormalTok{)}
\NormalTok{    \{}
\NormalTok{      wins }\OtherTok{=}\NormalTok{ wins}\SpecialCharTok{+}\DecValTok{1}
\NormalTok{    \}}
    \ControlFlowTok{if}\NormalTok{(sim[i]}\SpecialCharTok{!=}\DecValTok{6}\NormalTok{)}
\NormalTok{    \{}
\NormalTok{      loss }\OtherTok{=}\NormalTok{ loss }\SpecialCharTok{+} \DecValTok{1}
\NormalTok{    \}}
\NormalTok{  \}}
  
  
  \FunctionTok{print}\NormalTok{(}\StringTok{"Number of wins:"}\NormalTok{)}
  \FunctionTok{print}\NormalTok{(wins)}
  \FunctionTok{print}\NormalTok{(}\StringTok{"Losses"}\NormalTok{)}
  \FunctionTok{print}\NormalTok{(loss)}
  \FunctionTok{print}\NormalTok{(}\StringTok{"Win/Loss Ratio:"}\NormalTok{)}
  \FunctionTok{print}\NormalTok{(wins}\SpecialCharTok{/}\NormalTok{loss)}
  
\NormalTok{\}}
\end{Highlighting}
\end{Shaded}

\begin{Shaded}
\begin{Highlighting}[]
\FunctionTok{deMere1WinProb}\NormalTok{(}\DecValTok{50000}\NormalTok{)}
\end{Highlighting}
\end{Shaded}

\begin{verbatim}
## [1] "Number of wins:"
## [1] 8427
## [1] "Losses"
## [1] 41573
## [1] "Win/Loss Ratio:"
## [1] 0.2027037
\end{verbatim}

\begin{Shaded}
\begin{Highlighting}[]
\CommentTok{\#The proportion is less than 0.5}
\end{Highlighting}
\end{Shaded}

\#c.~de Mere also was interested in a game where he bet that in 24
independent rolls of a pair of fair dice, at least one of the 24 would
result in two sixes. Write a function deMere2 that returns a 1 if de
Mere wins, and returns a 0 if de Mere loses. When you use sample() you
might want to sample 24 times from c(0,1), with the probability of
getting a 1 being the probability of rolling a pair of sixes with two
fair dice. That way you can check whether de Mere wins by just seeing
whether the sum of the output of sample() is positive or not.

\begin{Shaded}
\begin{Highlighting}[]
\NormalTok{deMere2 }\OtherTok{=} \ControlFlowTok{function}\NormalTok{()}
\NormalTok{\{}
\NormalTok{  sim }\OtherTok{=} \FunctionTok{sample}\NormalTok{(}\FunctionTok{c}\NormalTok{(}\DecValTok{0}\NormalTok{,}\DecValTok{1}\NormalTok{),}\AttributeTok{size=}\DecValTok{24}\NormalTok{,}\AttributeTok{replace=}\ConstantTok{TRUE}\NormalTok{, }\AttributeTok{prob =} \FunctionTok{c}\NormalTok{(}\DecValTok{35}\SpecialCharTok{/}\DecValTok{36}\NormalTok{,}\DecValTok{1}\SpecialCharTok{/}\DecValTok{36}\NormalTok{))}
  \FunctionTok{print}\NormalTok{(}\StringTok{"Number of wins:"}\NormalTok{)}
  \FunctionTok{print}\NormalTok{(}\FunctionTok{sum}\NormalTok{(sim))}
\NormalTok{\}}
\end{Highlighting}
\end{Shaded}

\begin{Shaded}
\begin{Highlighting}[]
\FunctionTok{deMere2}\NormalTok{()}
\end{Highlighting}
\end{Shaded}

\begin{verbatim}
## [1] "Number of wins:"
## [1] 0
\end{verbatim}

\#d.~Now write a function called deMere2WinProb that takes as input a
number of repetitions numRep and which simulates numRep plays of the
game, returning the proportion in which de Mere wins. Run this function
with numRep equal to 50000 and report the proportion of games in which
de Mere wins. Is the proportion more or less than 0.5?

\begin{Shaded}
\begin{Highlighting}[]
\NormalTok{deMere2WinProb }\OtherTok{=} \ControlFlowTok{function}\NormalTok{(numRep)}
\NormalTok{\{}
\NormalTok{  sim }\OtherTok{=} \FunctionTok{sample}\NormalTok{(}\FunctionTok{c}\NormalTok{(}\DecValTok{0}\NormalTok{,}\DecValTok{1}\NormalTok{),}\AttributeTok{size=}\NormalTok{numRep,}\AttributeTok{replace=}\ConstantTok{TRUE}\NormalTok{, }\AttributeTok{prob =} \FunctionTok{c}\NormalTok{(}\DecValTok{35}\SpecialCharTok{/}\DecValTok{36}\NormalTok{,}\DecValTok{1}\SpecialCharTok{/}\DecValTok{36}\NormalTok{))}
\NormalTok{  wins }\OtherTok{=} \DecValTok{0}
\NormalTok{  loss }\OtherTok{=} \DecValTok{0}
  
  \ControlFlowTok{for}\NormalTok{(i }\ControlFlowTok{in} \DecValTok{1}\SpecialCharTok{:}\NormalTok{numRep)}
\NormalTok{  \{}
    \ControlFlowTok{if}\NormalTok{(sim[i]}\SpecialCharTok{==}\DecValTok{1}\NormalTok{)}
\NormalTok{    \{}
\NormalTok{      wins }\OtherTok{=}\NormalTok{ wins }\SpecialCharTok{+} \DecValTok{1}
\NormalTok{    \}}
    \ControlFlowTok{if}\NormalTok{(sim[i]}\SpecialCharTok{==}\DecValTok{0}\NormalTok{)}
\NormalTok{    \{}
\NormalTok{      loss }\OtherTok{=}\NormalTok{ loss }\SpecialCharTok{+} \DecValTok{1}
\NormalTok{    \}}
\NormalTok{  \}}
  
  \FunctionTok{print}\NormalTok{(}\StringTok{"Number of losses:"}\NormalTok{)}
  \FunctionTok{print}\NormalTok{(loss)}
  \FunctionTok{print}\NormalTok{(}\StringTok{"Win/Loss Ratio:"}\NormalTok{)}
  \FunctionTok{print}\NormalTok{(wins}\SpecialCharTok{/}\NormalTok{loss)}
  \FunctionTok{print}\NormalTok{(}\StringTok{"Number of wins:"}\NormalTok{)}
  \FunctionTok{return}\NormalTok{(wins)}
\NormalTok{\}}
\end{Highlighting}
\end{Shaded}

\begin{Shaded}
\begin{Highlighting}[]
\FunctionTok{deMere2WinProb}\NormalTok{(}\DecValTok{50000}\NormalTok{)}
\end{Highlighting}
\end{Shaded}

\begin{verbatim}
## [1] "Number of losses:"
## [1] 48658
## [1] "Win/Loss Ratio:"
## [1] 0.02758025
## [1] "Number of wins:"
\end{verbatim}

\begin{verbatim}
## [1] 1342
\end{verbatim}

\begin{Shaded}
\begin{Highlighting}[]
\CommentTok{\#Proportion is less than 0.5}
\end{Highlighting}
\end{Shaded}

\#e. You may wonder why the number of repetitions was chosen to be
50000. Later in the text tools will be developed that allow choices of
the number of repetitions to be made in a principled way. For now, we
will just observe how the proportion of wins changes as the number of
repetitions grows. Use the deMere1 function to generate 50000 plays of
the game, storing the resulting sequence of 0s and 1s in an R object
called deMere1Wins.

\begin{Shaded}
\begin{Highlighting}[]
\NormalTok{deMere1Wins }\OtherTok{=} \FunctionTok{sample}\NormalTok{(}\FunctionTok{c}\NormalTok{(}\DecValTok{0}\NormalTok{,}\DecValTok{1}\NormalTok{),}\AttributeTok{size=}\DecValTok{50000}\NormalTok{,}\AttributeTok{replace=}\ConstantTok{TRUE}\NormalTok{,}\AttributeTok{prob=}\FunctionTok{c}\NormalTok{(}\DecValTok{5}\SpecialCharTok{/}\DecValTok{6}\NormalTok{,}\DecValTok{1}\SpecialCharTok{/}\DecValTok{6}\NormalTok{))}
\NormalTok{wins }\OtherTok{=} \DecValTok{0}
\NormalTok{wins\_counter }\OtherTok{=} \FunctionTok{c}\NormalTok{()}

\ControlFlowTok{for}\NormalTok{(i }\ControlFlowTok{in} \DecValTok{1}\SpecialCharTok{:}\DecValTok{50000}\NormalTok{)}
\NormalTok{\{}
  \ControlFlowTok{if}\NormalTok{(deMere1Wins[i]}\SpecialCharTok{==}\DecValTok{1}\NormalTok{)}
\NormalTok{  \{}
\NormalTok{    wins }\OtherTok{=}\NormalTok{ wins }\SpecialCharTok{+} \DecValTok{1}
\NormalTok{    wins\_counter }\OtherTok{=} \FunctionTok{append}\NormalTok{(wins\_counter,wins)}
\NormalTok{  \}}
  \ControlFlowTok{if}\NormalTok{(deMere1Wins[i]}\SpecialCharTok{==}\DecValTok{0}\NormalTok{)}
\NormalTok{  \{}
\NormalTok{    wins\_counter }\OtherTok{=} \FunctionTok{append}\NormalTok{(wins\_counter,}\DecValTok{0}\NormalTok{)}
\NormalTok{  \}}
\NormalTok{\}}
\end{Highlighting}
\end{Shaded}

\#f.~Plot the proportion of games won versus the number of games played,
using a line (e.g., type = ``l'' in the plot() function.) Start with the
tenth game, so don't plot the first 9 games.

\begin{Shaded}
\begin{Highlighting}[]
\NormalTok{x }\OtherTok{=} \DecValTok{1}\SpecialCharTok{:}\DecValTok{50000}

\CommentTok{\#Unsure if I plot x / wins and get this basic graph, or plot win\_counter}

\FunctionTok{plot}\NormalTok{(x,wins}\SpecialCharTok{/}\NormalTok{x,}\AttributeTok{type=}\StringTok{"l"}\NormalTok{)}
\end{Highlighting}
\end{Shaded}

\includegraphics{STT380-HW1_files/figure-latex/unnamed-chunk-17-1.pdf}

\#g. Repeat e. and f.~for the second (pairs of sixes in 24 tosses) game,
and add the plot of the proportion of games won versus the number of
games played to the existing plot, again throwing out the first 9 games.

\begin{Shaded}
\begin{Highlighting}[]
\NormalTok{deMere2Wins }\OtherTok{=} \FunctionTok{sample}\NormalTok{(}\FunctionTok{c}\NormalTok{(}\DecValTok{0}\NormalTok{,}\DecValTok{1}\NormalTok{),}\AttributeTok{size=}\DecValTok{24}\NormalTok{,}\AttributeTok{replace=}\ConstantTok{TRUE}\NormalTok{, }\AttributeTok{prob =} \FunctionTok{c}\NormalTok{(}\DecValTok{35}\SpecialCharTok{/}\DecValTok{36}\NormalTok{,}\DecValTok{1}\SpecialCharTok{/}\DecValTok{36}\NormalTok{))}

\NormalTok{wins }\OtherTok{=} \DecValTok{0}
\NormalTok{wins\_counter }\OtherTok{=} \FunctionTok{c}\NormalTok{()}

\ControlFlowTok{for}\NormalTok{(i }\ControlFlowTok{in} \DecValTok{1}\SpecialCharTok{:}\DecValTok{24}\NormalTok{)}
\NormalTok{\{}
  \ControlFlowTok{if}\NormalTok{(deMere2Wins[i]}\SpecialCharTok{==}\DecValTok{1}\NormalTok{)}
\NormalTok{  \{}
\NormalTok{    wins }\OtherTok{=}\NormalTok{ wins }\SpecialCharTok{+} \DecValTok{1}
\NormalTok{    wins\_counter }\OtherTok{=} \FunctionTok{append}\NormalTok{(wins\_counter,wins)}
\NormalTok{  \}}
  \ControlFlowTok{if}\NormalTok{(deMere2Wins[i]}\SpecialCharTok{==}\DecValTok{0}\NormalTok{)}
\NormalTok{  \{}
\NormalTok{    wins\_counter }\OtherTok{=} \FunctionTok{append}\NormalTok{(wins\_counter,}\DecValTok{0}\NormalTok{)}
\NormalTok{  \}}
\NormalTok{\}}
\end{Highlighting}
\end{Shaded}

\begin{Shaded}
\begin{Highlighting}[]
\NormalTok{x }\OtherTok{=} \DecValTok{1}\SpecialCharTok{:}\DecValTok{24}

\CommentTok{\#same problem as the first graph}

\FunctionTok{plot}\NormalTok{(x,wins}\SpecialCharTok{/}\NormalTok{x,}\AttributeTok{type=}\StringTok{"l"}\NormalTok{)}
\end{Highlighting}
\end{Shaded}

\includegraphics{STT380-HW1_files/figure-latex/unnamed-chunk-19-1.pdf}

\#QUESTION 5: Exercise 1.8. (Adapted from Grinstead and Snell (2012))

\#Two people, Luke and Leia, are playing a coin-flipping game. A fair
coin is flipped 𝑛 times. If the coin comes up HEADS on a flip, Luke wins
1 dollar from Leia. If the coin comes up TAILS, Leia wins 1 dollar from
Luke. For example with 𝑛 = 10, if the sequence of tosses is

\#𝐻𝑇 𝑇 𝐻𝐻𝐻𝑇 𝐻𝑇 𝐻

\#then Luke's winnings at each stage would be {[}1, 0, −1, 0, 1, 2, 1,
2, 1, 2{]} with Luke being 2 dollars ahead at the end of the game.

\#We will investigate two questions related to this game with 𝑛 = 40. We
will focus on Luke's winnings, but the same analysis could be done for
Leia's winnings.

\#a. At the end of the game, Luke will have won some amount between −40
dollars(if all the tosses were TAILS) and 40 dollars (if all the tosses
were HEADS). Letting 𝑊 stand for Luke's winnings, what would you expect
the most likely value for 𝑊 to be? What would you expect the probability
mass function for 𝑊 to look like?

\begin{Shaded}
\begin{Highlighting}[]
\CommentTok{\#The most likely value for W would be somewhere between {-}20 and 20 dollars}

\CommentTok{\#The probability mass function for W would be (1/2), with decreasing odds by 1/2 chance up to, 40; So in both directions, a very small number}
\end{Highlighting}
\end{Shaded}

\#b. Write a function called lukeWinnings that takes as input a number
of repetitions numRep and that returns the simulated probability mass
function of Luke's winnings. So for example the function should return,
for 𝑊 = 2, the proportion of the repetitions that led to Luke winning 2
dollars at the end of the 40 flips.

\begin{Shaded}
\begin{Highlighting}[]
\NormalTok{lukeWinnings }\OtherTok{=} \ControlFlowTok{function}\NormalTok{(numRep,W) }\CommentTok{\#Taking in number of reps and endgame}
\NormalTok{\{}
  \FunctionTok{return}\NormalTok{(}\FunctionTok{dbinom}\NormalTok{(W,}\DecValTok{1}\SpecialCharTok{:}\NormalTok{numRep,.}\DecValTok{5}\NormalTok{))}
\NormalTok{\}}
\end{Highlighting}
\end{Shaded}

\begin{Shaded}
\begin{Highlighting}[]
\FunctionTok{lukeWinnings}\NormalTok{(}\DecValTok{40}\NormalTok{,}\DecValTok{2}\NormalTok{)}
\end{Highlighting}
\end{Shaded}

\begin{verbatim}
##  [1] 0.000000e+00 2.500000e-01 3.750000e-01 3.750000e-01 3.125000e-01
##  [6] 2.343750e-01 1.640625e-01 1.093750e-01 7.031250e-02 4.394531e-02
## [11] 2.685547e-02 1.611328e-02 9.521484e-03 5.554199e-03 3.204346e-03
## [16] 1.831055e-03 1.037598e-03 5.836487e-04 3.261566e-04 1.811981e-04
## [21] 1.001358e-04 5.507469e-05 3.015995e-05 1.645088e-05 8.940697e-06
## [26] 4.842877e-06 2.615154e-06 1.408160e-06 7.562339e-07 4.051253e-07
## [31] 2.165325e-07 1.154840e-07 6.146729e-08 3.265450e-08 1.731678e-08
## [36] 9.167707e-09 4.845788e-09 2.557499e-09 1.347871e-09 7.094059e-10
\end{verbatim}

\#This is somewhat similar to the simulation done earlier in this
chapter that returned the simulated probability mass function of the
number of HEADS in 25 flips, so you might want to return to that example
if desired.

\#c.~Graph the simulated probability mass function, using vertical bars
to represent the probabilities (e.g., type = ``h'' in the plot()
function). Does the probability mass function agree with your
expectations from a.?

\begin{Shaded}
\begin{Highlighting}[]
\NormalTok{x }\OtherTok{=} \DecValTok{1}\SpecialCharTok{:}\DecValTok{40}

\FunctionTok{plot}\NormalTok{(x,}\FunctionTok{lukeWinnings}\NormalTok{(}\DecValTok{40}\NormalTok{,}\DecValTok{2}\NormalTok{),}\AttributeTok{type=}\StringTok{\textquotesingle{}h\textquotesingle{}}\NormalTok{)}
\end{Highlighting}
\end{Shaded}

\includegraphics{STT380-HW1_files/figure-latex/unnamed-chunk-23-1.pdf}

\#d.~Next we turn our attention to the amount of times during the 40
tosses that Luke is in the lead. Most peoples' intuition suggests that
out of 40 plays, since the game is fair, the most likely number of times
that Luke would be in the lead would be 20, and it would be quite
unlikely that Luke would be in the lead either 0 or 40 times. Does that
agree with your expectations?

\begin{Shaded}
\begin{Highlighting}[]
\CommentTok{\#Yes, that does agree with my expectations}
\end{Highlighting}
\end{Shaded}

\#e. First, suppose that an R vector tmp contains the results of one
play of the game, with a value of 0 representing TAILS and a value of 1
representing HEADS. For example, if the first 10 tosses were

\#𝐻𝑇 𝑇 𝐻𝐻𝐻𝑇 𝐻𝑇 𝐻

\#then the vector tmp would contain 1, 0, 0, 1, 1, 1, 0, 1, 0, 1 in its
first 10 places. Verify that the R code 2*cumsum(tmp) - 1:40 returns
Luke's winnings at each play of the game.

\begin{Shaded}
\begin{Highlighting}[]
\NormalTok{e\_coin }\OtherTok{=} \FunctionTok{c}\NormalTok{(}\DecValTok{0}\NormalTok{,}\DecValTok{1}\NormalTok{)}

\NormalTok{game }\OtherTok{=} \FunctionTok{c}\NormalTok{(}\FunctionTok{sample}\NormalTok{(e\_coin,}\AttributeTok{size=}\DecValTok{40}\NormalTok{,}\AttributeTok{replace =} \ConstantTok{TRUE}\NormalTok{,}\AttributeTok{prob=}\FunctionTok{c}\NormalTok{(}\DecValTok{1}\SpecialCharTok{/}\DecValTok{2}\NormalTok{,}\DecValTok{1}\SpecialCharTok{/}\DecValTok{2}\NormalTok{)))}
\NormalTok{tmp }\OtherTok{=} \FunctionTok{c}\NormalTok{()}

\ControlFlowTok{for}\NormalTok{(i }\ControlFlowTok{in} \DecValTok{1}\SpecialCharTok{:}\FunctionTok{length}\NormalTok{(game))}
\NormalTok{\{}
  \ControlFlowTok{if}\NormalTok{(game[i]}\SpecialCharTok{==}\DecValTok{1}\NormalTok{)}
\NormalTok{  \{}
\NormalTok{    tmp }\OtherTok{=} \FunctionTok{append}\NormalTok{(tmp,}\DecValTok{1}\NormalTok{)}
\NormalTok{  \}}
  \ControlFlowTok{if}\NormalTok{(game[i]}\SpecialCharTok{==}\DecValTok{0}\NormalTok{)}
\NormalTok{  \{}
\NormalTok{    tmp }\OtherTok{=} \FunctionTok{append}\NormalTok{(tmp,}\DecValTok{0}\NormalTok{)}
\NormalTok{  \}}
\NormalTok{\}}

\NormalTok{winnings }\OtherTok{=} \DecValTok{2}\SpecialCharTok{*}\FunctionTok{cumsum}\NormalTok{(tmp) }\SpecialCharTok{{-}} \DecValTok{1}\SpecialCharTok{:}\DecValTok{40}

\FunctionTok{print}\NormalTok{(}\StringTok{\textquotesingle{}values of tmp:\textquotesingle{}}\NormalTok{)}
\end{Highlighting}
\end{Shaded}

\begin{verbatim}
## [1] "values of tmp:"
\end{verbatim}

\begin{Shaded}
\begin{Highlighting}[]
\FunctionTok{print}\NormalTok{(tmp)}
\end{Highlighting}
\end{Shaded}

\begin{verbatim}
##  [1] 0 0 1 0 0 0 0 1 1 0 1 0 0 1 1 0 0 1 0 0 1 0 0 0 0 0 1 0 1 0 1 1 0 1 1 1 1 0
## [39] 1 1
\end{verbatim}

\begin{Shaded}
\begin{Highlighting}[]
\FunctionTok{print}\NormalTok{(}\StringTok{\textquotesingle{}\textquotesingle{}}\NormalTok{)}
\end{Highlighting}
\end{Shaded}

\begin{verbatim}
## [1] ""
\end{verbatim}

\begin{Shaded}
\begin{Highlighting}[]
\FunctionTok{print}\NormalTok{(}\StringTok{\textquotesingle{}winnings:\textquotesingle{}}\NormalTok{)}
\end{Highlighting}
\end{Shaded}

\begin{verbatim}
## [1] "winnings:"
\end{verbatim}

\begin{Shaded}
\begin{Highlighting}[]
\FunctionTok{print}\NormalTok{(winnings)}
\end{Highlighting}
\end{Shaded}

\begin{verbatim}
##  [1]  -1  -2  -1  -2  -3  -4  -5  -4  -3  -4  -3  -4  -5  -4  -3  -4  -5  -4  -5
## [20]  -6  -5  -6  -7  -8  -9 -10  -9 -10  -9 -10  -9  -8  -9  -8  -7  -6  -5  -6
## [39]  -5  -4
\end{verbatim}

\#f.~Write a function called leadsDist that takes a number of
repetitions numRep as input, and returns the simulated probability mass
function of the number of times, during the 40 tosses, that Luke is in
the lead. You will probably find e. above to be helpful.

\begin{Shaded}
\begin{Highlighting}[]
\NormalTok{leadsDist }\OtherTok{=} \ControlFlowTok{function}\NormalTok{(numRep)}
\NormalTok{\{}
\NormalTok{  e\_coin }\OtherTok{=} \FunctionTok{c}\NormalTok{(}\DecValTok{0}\NormalTok{,}\DecValTok{1}\NormalTok{)}

\NormalTok{  game }\OtherTok{=} \FunctionTok{c}\NormalTok{(}\FunctionTok{sample}\NormalTok{(e\_coin,}\AttributeTok{size=}\NormalTok{numRep,}\AttributeTok{replace =} \ConstantTok{TRUE}\NormalTok{,}\AttributeTok{prob=}\FunctionTok{c}\NormalTok{(}\DecValTok{1}\SpecialCharTok{/}\DecValTok{2}\NormalTok{,}\DecValTok{1}\SpecialCharTok{/}\DecValTok{2}\NormalTok{)))}
\NormalTok{  tmp }\OtherTok{=} \FunctionTok{c}\NormalTok{()}
\NormalTok{  times\_in\_lead }\OtherTok{=} \DecValTok{0}

  \ControlFlowTok{for}\NormalTok{(i }\ControlFlowTok{in} \DecValTok{1}\SpecialCharTok{:}\FunctionTok{length}\NormalTok{(game))}
\NormalTok{  \{}
    \ControlFlowTok{if}\NormalTok{(game[i]}\SpecialCharTok{==}\DecValTok{1}\NormalTok{)}
\NormalTok{    \{}
\NormalTok{      tmp }\OtherTok{=} \FunctionTok{append}\NormalTok{(tmp,}\DecValTok{1}\NormalTok{)}
      \ControlFlowTok{if}\NormalTok{(tmp[i]}\SpecialCharTok{\textgreater{}}\DecValTok{0}\NormalTok{)}
\NormalTok{      \{}
\NormalTok{        times\_in\_lead }\OtherTok{=}\NormalTok{ times\_in\_lead }\SpecialCharTok{+} \DecValTok{1}
\NormalTok{      \}}
\NormalTok{    \}}
    \ControlFlowTok{if}\NormalTok{(game[i]}\SpecialCharTok{==}\DecValTok{0}\NormalTok{)}
\NormalTok{    \{}
\NormalTok{      tmp }\OtherTok{=} \FunctionTok{append}\NormalTok{(tmp,}\DecValTok{0}\NormalTok{)}
\NormalTok{    \}}
\NormalTok{  \}}
  \FunctionTok{print}\NormalTok{(}\StringTok{\textquotesingle{}times in lead:\textquotesingle{}}\NormalTok{)}
  \FunctionTok{print}\NormalTok{(times\_in\_lead)}
  \FunctionTok{return}\NormalTok{(}\FunctionTok{dbinom}\NormalTok{(}\DecValTok{1}\SpecialCharTok{:}\NormalTok{times\_in\_lead,numRep,}\AttributeTok{p=}\DecValTok{1}\SpecialCharTok{/}\DecValTok{2}\NormalTok{))}
\NormalTok{\}}
\end{Highlighting}
\end{Shaded}

\begin{Shaded}
\begin{Highlighting}[]
\NormalTok{value }\OtherTok{=} \FunctionTok{leadsDist}\NormalTok{(}\DecValTok{40}\NormalTok{)}
\end{Highlighting}
\end{Shaded}

\begin{verbatim}
## [1] "times in lead:"
## [1] 20
\end{verbatim}

\begin{Shaded}
\begin{Highlighting}[]
\NormalTok{value}
\end{Highlighting}
\end{Shaded}

\begin{verbatim}
##  [1] 3.637979e-11 7.094059e-10 8.985808e-09 8.311872e-08 5.984548e-07
##  [6] 3.490986e-06 1.695622e-05 6.994440e-05 2.486912e-04 7.709428e-04
## [11] 2.102571e-03 5.081214e-03 1.094415e-02 2.110658e-02 3.658474e-02
## [16] 5.716365e-02 8.070163e-02 1.031187e-01 1.194007e-01 1.253707e-01
\end{verbatim}

\#g. Graph the simulated probability mass function, using vertical bars
to represent the probabilities (e.g., type = ``h'' in the plot()
function). Does the probability mass function agree with your
expectations from d.?

\begin{Shaded}
\begin{Highlighting}[]
\NormalTok{x }\OtherTok{=} \FunctionTok{length}\NormalTok{(value)}

\FunctionTok{plot}\NormalTok{(}\DecValTok{1}\SpecialCharTok{:}\NormalTok{x,value,}\AttributeTok{type=}\StringTok{\textquotesingle{}h\textquotesingle{}}\NormalTok{)}
\end{Highlighting}
\end{Shaded}

\includegraphics{STT380-HW1_files/figure-latex/unnamed-chunk-28-1.pdf}

\begin{Shaded}
\begin{Highlighting}[]
\CommentTok{\#A majority of the time, it does agree. Though there are some runs where it breaks this assumption}
\end{Highlighting}
\end{Shaded}


\end{document}
